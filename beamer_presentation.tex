% Small example for a beamer presentation
\documentclass{beamer}

%\usepackage[utfx8]{inputenc}
\usepackage{default}

%\usepackage[T1]{fontenc}
%\usepackage[latin1]{inputenc}
\usepackage[ngerman]{babel}

\title{Simplex-Algorithmus}
\author{Selman, Laing, U\c{c}ar, Neumann}
\date{\today}

\begin{document}
\begin{frame}
\frametitle{Probleme bei der Implementierung und Auswertung} 
\"Uber folgende Fragestellungen m\"ussen wir noch nachdenken:
\\
\pause
\begin{enumerate}
 \item<1-> Datenkommunikation
 \pause
 \\
  \begin{enumerate}
   \item<1-> externe Dateien als Eingabedaten: Welche Dateiformate? (z.B.: *.xls; *.txt; *.m)
   \pause
   \item<2-> Welche built-in-Funktionalit\"aten bietet Matlab f\"ur 1.1? 
   \pause
   \item<3-> Aufbereitung der Ausgabedaten: Wie soll das Simplex-Tableau genau aussehen?
   \pause
  \end{enumerate}
 \item<2-> Verhalten des Programms in Fehlersituationen
 \pause
  \begin{enumerate}
   \item<1-> Welche Datentypen sollen als Koeffizienten akzeptiert werden? \\ ($\rightarrow$ Mantissenl\"ange?)
             Inwieweit soll das Programm m\"ogliche \"Uberlauf-Situationen (u.\"A.) abfangen? 
   \pause
   \item<2-> Welche weiteren Fehlersituationen können auftreten?
  \end{enumerate}
\end{enumerate}
\end{frame}

%\begin{frame}[allowframebreaks]
%\frametitle{erster Entwurf} 
%\includegraphics[scale=0.5]{screen.png} \\
%\includegraphics[scale=0.5]{screen0.png} \\
%\includegraphics[scale=0.35]{screen1.png}
%\end{frame}

\end{document}